%!TEX root=main.tex

\section{Futuras Melhorias}

Relativamente à componente de IA existem algoritmos e abordagens não exploradas. A implementação de ~\verb|iterative deepening|  em vez da realização de uma pesquisa única seria um ponto simples de implementar na continuação do projeto. A implementação de ~\verb|negaScout| e a ordenação de tabuleiros por uma ordem não aleatória baseada numa classificação de modo a facilitar a poda também seriam pontos interessantes a explorar.

Face à mini \verb|framework| implementada, seria interessante completá-la refinando e generalizando o PentagoPandora de modo a que consiga criar a árvore Minmax para qualquer jogo e para qualquer jogador.

O interface gráfico e a implementação em Unity3d pode ser ainda melhorada em diversos pontos, seja na apresentação, seja no código implementado. Com o intuito de disponibilizar parte do projeto na Asset Store do Unity3d, todo  grafismo presente não original deverá ser substituído e o código do projeto será revisto após a entrega do trabalho sendo que deverá ficar disponível para download antes de Outubro, caso passe nos testes de qualidade e não existam restrições legais que impeçam a publicação.





