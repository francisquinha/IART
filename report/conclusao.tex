%!TEX root=main.tex

\section{Conclusões}

Quando se pretende implementar um jogador artificial que recorra a algoritmos minmax, a maior dificuldade será, em princípio, conseguir uma boa heurística para avaliar os tabuleiros de jogo. Essa foi também uma das maiores dificuldades que encontramos neste projeto. De facto, para cada heurística baseada numa ideia original, as primeiras tentativas apresentavam defeitos graves de performance contra humanos. Mesmo contra a heurística de controlo, que escolhe um valor aleatório para o tabuleiro, os resultados não eram muito animadores. Foi necessário observarmos vários jogos concretos para percebermos o que devíamos mudar nas nossas heurísticas. No final, consideramos que, neste aspeto, o projeto se encontra sólido, dado que a heurística A star apresenta um comportamento bastante satisfatório contra jogadores humanos. Esta também foi a heurística que melhor se portou em geral nos testes automáticos de performance.

Além das heurísticas, a outra grande parte deste tipo de projetos é a escolha das jogadas seguintes. Na sua versão mais ingénua, o minmax percorre todas as jogadas seguintes à procura da mais adequada. No entanto, isso muitas vezes pode ser melhorado. Não apenas no sentido de usar cortes alfa-beta ou cortes devidos a considerarmos um determinado resultado suficientemente bom. Mas no sentido de perceber sem análise dos descendentes que um determinado tabuleiro conduzirá aos mesmos resultados que outro previamente analisado. Esse é precisamente o caso do Pentago em que as simetrias desempenham um papel fundamental. Apesar de ser desde sempre óbvio que este era um tema importante para explorar, demorou algum tempo a perceber como o fazer. Estamos convencidos que o resultado final é bastante interessante, nomeadamente o facto de nem sempre compensar esta verificação de simetrias.